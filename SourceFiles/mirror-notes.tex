

\subsection{Non-euclidean setting: mirror descent -- Jeffrey}
The dimension-free oracle bounds that have been discussed up until this point rely upon a well-behaved objective function $f$ and constraint set $\mathcal{X}$ in the $l_2$ norm; namely, the radius of the constraint set $R$ and Lipschitz constant of the objective function $L$ with respect to the $l_2$ norm must be independent of the ambient dimension $n$ in order to achieve a dimension-free oracle bound. In some settings, $f$ and $\mathcal{X}$ may be well-behaved with respect to an arbitrary norm $\|\cdot \|$, in which we wish to exploit the underlying structure to achieve fast convergence rates independent of the ambient dimension $n$. In this section, we provide an overview of the details of mirror descent, a method developed by \cite{blair1985problem}, which allows us to achieve a faster convergence rate for vector spaces endowed with an arbitrary norm. 


\subsection{Dual space}
connections with Fenchel, etc
- Gradient lives dual space
- In the case of Hilbert S

\subsection{Mirror map}
- Properties of Mirror map
- What this induces for the purposes of mirror descent

\subsection{Bregman divergences}
 - Bregman divergence definition
 - Can be interpreted as the difference between the function at x and the first-order taylor expansion around y for a given potential function
 - Generalization of squared Euclidean norm
 - Bregman is the law of cosines of FOC (?)
 - Duality
 - (Ashia paper) The Bregman Lagrangian is an approximation of the Hessian but the Hessian is difficult to discretize

Examples:
- KL divergence
- $L_P$ norm
- Mutual Information/hamming distance etc

\subsection{Mirror descent}
- Mirror descent and convergence results
- Examples of varying setups
- Equivalent in different norms but might introduce an ambient dimension
Adaboost connection to the entropy?

