

\subsection{Intro, Background, References}
Considering problems like
\[ \min_{x\in\mathcal{X}} f(x) \]
for convex $f$ and $\mathcal{X}$. First order methods: normal gradient descemt methods can easily be interpreted as an ODE. Then ODE theory provides guidance in algorithm design -- elegant Lyapunov arguments in continuous time analogous for discretizations. Here we combine Nesterov's accelerated method [12] with Nemirovski's mirror descent [19] (19 may be a better source on mirror descent than the book chapter). Nesterov's method applied in [23, 20, 21, 4]. Present an ODE showing that a broad family of accelerated methods can be obtained from discretization of a simple ODE (superset for Candes paper).

\subsection{Relevant Sources}
\begin{enumerate}
    \item Nemirovski's mirror descent [19]
    \item applications of acceleration [23, 20, 21, 4] (ideas for experiments and stuffing for the intro of our paper)
    \item Ch 11.2 in [11] or Appendix A in [2] for Bregman
    \item forward/backward Euler scheme in Ch 2 [10]
\end{enumerate}

\subsection{Lyapunov Argument}
The convergence rates for ODEs $X(t)$ and discrete time algorithms $X_k$ are analyzed by an argument involving a Lyapunov function $V(X)$ which must be nonnegative and decreasing wrt $t$. For example, in normal gradient descent we have $\dot X(t) = -\nabla f(X(t))$, $V(
X) = \frac{1}{2}\|X(t)-x^*\|^2$
\[\dot V(X(t)) = \langle  -\nabla f(X(t)), X(t) - x^*\rangle \leq -(f(X(t)) - f^*)\]
\[ V(X(t)) - V(x_0) \leq tf^* - \int_0^t f(X(t))dt \leq tf^* - f\left(\int_0^t X(t)dt\right) \implies  f^* - f\left(\frac{1}{t}\int_0^t X(t)dt\right) \leq \frac{V(x_0)}{t} \]
Thus the general method is to find a Lyapunov function, then integrate and rearrange to find a convergence rate.

\subsection{Mirror Descent}
Consider the Lyapunov function as a generator for these first order methods -- choose $V$ and then design dynamics of $X$. Replace the $V(X)$ above with a function on the dual space for $Z$, $X(t) = \nabla\psi^*(Z(t))$, $\psi$ convex
\[V(Z(t)) = D_{\psi^*} (Z(t), z^*)\]
This is the Bregman divergence: 
\[ D_{\psi^*}(z,y) = \psi^*(z) - \psi^*(y) - \langle \psi^*(y), z-y\rangle \]
Can have properties such as \textit{l} strongly convex or $L$ smooth. We find that
\[\dot V(Z(t)) = \langle  X(t) - x^*, \dot Z(t)\rangle\]
So $\dot Z = -\nabla f(X)$ follows argument above, recovering mirror descent, defined (with consisdent starting point)
\[X = \nabla\psi^*(Z), \quad \dot Z = -\nabla f(X)\]

\section{Continuous Time Accelerated Mirror Descent}
We can view Candes paper as revolving around the Lyapunov function 
\[\mathcal{E}(t) = \frac{t^2}{r} (f(X) - f^*) + \frac{r}{2} \|X+\frac{t}{r}\dot X - x^* \|^2\]

\subsection{Derivation of continuous-time mirror descent}
We propose the function
\[ V(X(t),Z(t),t) = \frac{t^2}{r} (f(X(t)) - f^*) + r D_{\phi^*} (Z(t) z^*) \]
If we take the time derivative and chose dynamics $\dot Z = -\frac{t}{r} \nabla f(X)$, and also $\nabla \psi^*(Z) = X + \frac{t}{r} \dot X$ $\nabla \psi^*(z^*) = x^*$ (*** intuitively what can we say about $\psi^*$)
\[ \frac{d}{dt} V(X(t),Z(t),t) \leq -t \frac{r-2}{r} (f(X) - f^*) \]
Then $V$ is Lyapunov for $r\geq 2$. Proposed system, for $X(0) = x_0 = \nabla \psi^*(z_0), Z(0) = z_0$
\[ \dot X = \frac{r}{t} (\nabla \psi^*(Z) - X),\quad \dot Z = -\frac{t}{r} \nabla f(X) \]

\subsection{Continuous-time averaging interpretation}
In integral form, we can write
\[ X(t) = \frac{\int_0^t \tau^{r-1} \nabla\psi^*(Z(\tau)) d\tau}{\int_0^t \tau^{r-1}} \]
So the dual variable $Z$ accumulates gradients with a $\frac{t}{r}$ rate, and primal $X$ is a weighted average of $\nabla\psi^*(Z)$, with weights determined by $t^{r-1}$. Here, it is clear that the primal trajectory remains in $\mathcal X$ because it is convex and $\nabla\psi^*$ maps into $\mathcal X$.

\subsection{ODE solution sketch}
The Cauchy-Lipshitze existance and uniqueness theorem does not apply because of the singularity at $t=0$, so consider a sequence of approximating ODEs instead, as in Candes paper. Basically, replace $t$ in the denominator with $\max (t,\delta)$ In this process, we assume that $\nabla f$ is $L_f$-Lipschitz. and $\psi^*$ is $L_{\psi^*}$ smooth or equivalently $\nabla \psi^*$ is $L_{\psi^*}$-Lipschitz.

\subsection{Convergence rate}
By construction of the Lyapunov function, easy to see that
\[ \frac{t^2}{r} (f(X(t)) - f^*) \leq V(X(t),Z(t),t) \leq  V(x_0,z_0,0)  = rD_{\psi^*}(z_0,z^*) \]

\section{Discretization}
\subsection{Euler scheme}
Mixed forward/backward Euler scheme using step size $\sqrt{s}$ so $t_k = k\sqrt{s}$. (why $k+1$ on the $x$?)
\[ \frac{x_{k+1} - x_k}{\sqrt{s}} = \frac{r}{k\sqrt{s}} (\nabla \psi^*(z_k) - x_{k+1}),\quad \frac{z_{k+1} - z_k}{\sqrt{s}} = -\frac{k\sqrt{s}}{r} \nabla f(x_{k+1}) \]
Simplifying, we get (with $\lambda_k = \frac{r}{r+k}$)
\[ x_{k+1}  = \lambda_k \nabla \psi^*(z_k) + (1-\lambda_k) x_{k}, \quad z_{k+1} = z_k -\frac{ks}{r} \nabla f(x_{k+1}) \]
If we do something special ** with $\psi$ and $\psi^*$ we can recover that for $\tilde z_{k+1} = \nabla \psi^*(z_{k+1} )$
\[ \tilde z_{k+1} = \argmin_{x\in\mathcal{X}} \frac{ks}{r} \langle \nabla f(x_{k+1}), x \rangle + D_\psi (x, \tilde z_k)\]

Then to analyze, start with analogous potential function
\[E_k = V(x_k, z_k, k\sqrt{s}) = \frac{k^2s}{r} (f(x_k) - f^*) + rD_{\psi^*}(z_k,z^*)\]
Through some manipulations, we find that there is an extra term that prevents us from concluding that it is a Lyapunov function. (** with numerical experiments, would this descretized version work?)
\[ E_{k+1} - E_k \leq -\frac{s[(r-2)k-1]}{r} (f(x_{k+1}) - f^*) + rD_{\psi^*}(z_{z+1},z^*) \]

\subsection{Proposed discretization}
We make an alteration such that in the $x$ update, we replace $x_k$ with 
\[\tilde x_k = \argmin_{x\in\mathcal X} \gamma s \langle \nabla f(x_k), x \rangle + R(x,x_k)  \]
for a regularization function $R$ *** extend: are there other functions $R$ we could try other than what was proposed here: a distance function $D_\phi (x,x')$. It must be true that $\frac{l_R}{r} \|x-x'\|^2 \leq R(x,x') \leq \frac{L_R}{2} \|x-x'\|^2$
Now $x_{k+1}$ is a combination of a gradient step and a mirror descent update.

... type in full scheme

Then the consistency of this discretization with an extra step holds because $\tilde x_k = x_k + O(s)$

The convergence rate is then shown for an energy function $\tilde E_k = V(\tilde x_k, z_k, k\sqrt{s})$. Should take a closer look at this proof to better understand $\tilde x_k$.

Conditions on this convergence crop up for $\gamma$ and $s$

\section{Example and Numerical Experiments}
\subsection{Entropic descent: simplex constrained problems}
If we have a simplex constrained problem, we can take $\psi$ to be the negative entropy on a simplex. ***what are another analogous constraints that can be encoded in this way***

The function $\psi$ include $\delta$ to satisfy the constraint, and $\psi^*$ can be computed explicitly. Then mirror descent take $O(n)$ while $\phi^*$ is smooth for l$\infty$. Then take $R(x,y) = D_\phi(x,y)$ for $\phi$ a smoothed negative entropy function. $\nabla \phi^*$ can be computed in $O(n\log n)$, or $O(n)$ for radnomized alg. 

\subsection{Numerical Experiments}
Quadratic functoin and log-sum-exp on the simplex problem as above. *** what are applications of the simplex constraint**. Plots show accelerated, non-accelerated, as well as two faster algorithms with restarting mechanisms. 

Looks like there are cool videos in the supplementary material!






