\documentclass{article}
\usepackage[utf8]{inputenc}

\usepackage{amsmath}



\title{EE227BT Project:
Understanding the Continuous-Time Limit of Nesterov Acceleration}
\author{Nilesh Tripuraneni (nilesh\_tripuraneni@berkeley.edu, 3032089919)\\ Sarah Dean (sarahdean@eecs.berkeley.edu, 3031893242)\\ Jeffrey Chan (chanjed@berkeley.edu, 24988067)\\ Aldo Pacchiano (pacchiano@berkeley.edu, 26995108)}
\date{October 25, 2016}
\begin{document}
\maketitle


\section{Project}

This project will focus on understanding recent approaches to study the continuous-time limit of ``accelerated'' optimization algorithms \cite{su2014differential, wibisono2016variational, krichene2015accelerated}. After
the literature review we will hopefully be able to experiment with deriving different optimizers (using different numerical discretization schemes) of the continuous-time formulations.

\section{Background}
% Gradient descent update
First order methods for solving convex optimization problems in the form
\[\min f(x)\]
are popular for large data sets and problems common in fields like machine learning and compressed sensing. Gradient descent is a common such method, dating back to Euler and Lagrange. It takes the form of an update rule
\[x_{k+1} = x_k - s \nabla f(x_k)\]


% Nesterov's acceleration
Nesterov's acceleration gradient method is a first-order gradient method which may take the following form: starting with $x_0$ and $y_0 = x_0$ inductively define

\begin{align}
    x_k = y_{k-1} - s \nabla f(y_{k-1})\\
    y_k = x_k + \frac{k-1}{k+2} (x_k - x_{k-1})
\end{align}

For any fixed step size $s \leq \frac{1}{L}$ where $L$ is the Lipschitz constant of $\nabla f$, it has a convergence rate of:
 
\begin{equation}
f(x_k) - f^* \leq O \left(\frac{\|x_0 - x^*\|^2}{sk^2} \right)
\end{equation}
which is optimal among first order methods (cite wherever this was shown -- Nesterov, 2004)

% - notable applications can be found in sparse linear regression, compressed sensing, and deep/recurrent neural networks

%- Similar first order methods in non-Euclidean spaces

\bibliographystyle{unsrt}
\bibliography{refs}





\end{document}
